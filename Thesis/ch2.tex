%   MSc Business Analytics Dissertation
%
%   Title:     Aaa Bbbbbbb Cccccccccc
%   Author(s): Xxxxxx Xxxxxxxxx and Yyy Yyyyyyyyy
%
%   Chapter 2: Business Background
%
%   Change Control:
%   When     Who   Ver  What
%   -------  ----  ---  --------------------------------------------------------------
%   11Feb11  AB    0.1  Begun 
%

\chapter{Business Background}\label{C.Business.Background}

\section{Introduction}\label{S.intro2}
KPMG is one of the most renowned Big Four auditors and provides tax, audit, advisory and consultancy services to various clients. Information Risk Management is the service line of the organization that provides information systems security assurance while minimizing risks and frauds. For accuracy of financial reports, IT organizations depend on an effective audit. KPMG's IRM audit team works with clients and auditors to assist them to obtain their desired results; by assuring customers how IT functions are efficiently controlled and by ensuring auditors that their work is efficient and accurate within the guidelines. IRM audit team supports audit planning process and fraud risk assessment to monitor IT risks; supervises processes for a particular industry; supports auditors; assesses application controls design; supports testing phase of the whole audit process. Benefits of the services provided by IRM audit team are efficient and effective audits, impactful audit decisions and opinions, precise identification of business risks and issues reporting to senior management and audit committee.\\

\section{Business Contribution}
There has been a rapid loan growth since last few decades, which led to aggressive lending(weak controls and lenient standards). This increased lending can come from a volatile source. Auditing loan portfolios are imperative to make sure safety and compliance with regulatory requirements.The objective of auditing is to find errors and issues and take appropriate corrective measures or actions. Auditing of residential loan portfolios can alert users and banks about the deviations in prescribed policies of credit risks and therefore maintains sustainability and profits of banks. As mentioned in \ref{C.intro} since the Irish property bubble in 2007-2010, the focus has been increased on the performance of loan portfolios especially in residential sector to achieve:
\begin{itemize}
\item Interactive way to identify patterns in datasets to drill down into problem areas
\item Well timed potential issues indicators that adhere to provisions of audit processes and assessment of residential loans
\item Better and greater coverage of problem areas and increased focus on judgemental loan applications
\item Integration of useful and relevant market data and economic indicators for enhanced loan assessment
\end{itemize}

There has been a significant improvement in technology that helps in analyzing data interactively and graphically. Growth in financial services has led to increase in accuracy of loan data and better availability of external data sources. This practicum will bring together such information in an interactive way to enhance credit analysis, audit and assessment of residential loan portfolios to reduce the cost of credit analysis, enable faster credit decisions, close monitoring of accounts and prioritize collections.
 