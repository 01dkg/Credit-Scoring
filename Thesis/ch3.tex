%   MSc Business Analytics Dissertation
%
%   Title:     Aaa Bbbbbbb Cccccccccc
%   Author(s): Xxxxxx Xxxxxxxxx and Yyy Yyyyyyyyy
%
%   Chapter 3: Literature Review
%
%   Change Control:
%   When     Who   Ver  What
%   -------  ----  ---  --------------------------------------------------------------
%   11Feb11  AB    0.1  Begun 
%

\chapter{Literature Review}\label{C.LitReview}
\section{Introduction}\label{S.intro3}

In recent years, purchasing capability of an \textbf{economy} has increased due to improvement in their finances, and employment levels. Ranging from buying small household items to\textbf{ expensive items such as a house, a car or an office}. To buy a house or a car, one needs to have a large amount of money available to him; that is not necessarily possible most of the time. \\

\textbf{Start with an exmaple to explain what is loan}. There are certain critical circumstances that can occur anytime, where one may need a certain amount of cash. So one may need to borrow a generous amount of money from some other entity which is called a loan. A loan is lending a sum of money from one entity to another that involves repayment of the amount in near future. Lent amount is called principal amount and amount to be repaid is a summation of principal amount and an interest amount or other charges. It is not as easy as it sounds like, there are certain terms need to be agreed upon by each entity before exchange of the money. A loan can be for an amount taken at one time or can be taken in instalments {Partial Payments]. A loan can be provided by banks, corporations and financial institutions. Banks and financial institutions provide various types of loans as per the need of an applicant, such as personal loans, home loans, business loans, credit card loans and cash advances. There are times when the borrowing amount is very large and banks cannot provide the loan based on verbal agreement, they need to ensure that if an applicant is not able to repay the loan then they need to have a source to recover the lent amount. So, in this case, an applicant needs to apply for a mortgage with the bank.\\

A mortgage or collateral is an instrument that applicant has to pay back with predefined series of payments to the bank and financial institutions. Over a duration of time, an applicant needs to repay the loan inclusive of interest amount in order to free his/her mortgage. In case, if an applicant is not able to repay the loan within predetermined time, then the bank can recover their money by selling or putting it for auction the mortgage. The most common type of mortgage is residential mortgages were applicant gives his/her house to banks and in a case of no repayment then a bank will claim the house to recover the balance amount of the loan. This will give a bank a security that their lent amount is not at risk and over the years they will get back their lent money one way or the other. Mortgages come in various different forms. Most commonly used mortgage types are Fixed Rate Mortgage where applicant repays the loan amount on a fixed rate throughout the period determined and Adjustable Rate Mortgage where interest rate varies as per the changes in market interest rates. Our work is based on analysis of residential mortgages with varied interest types which will be discussed in later sections.\\

Put \textbf{Photo of loan application process folw chart}

Before analysing data based on residential mortgages, one needs to understand the process of giving a loan. Depending upon the requirement an applicant applies for a loan by filling an application form with all the necessary details required by the bank. Bank officials then analyse the application and may ask an applicant for additional information; after evaluation, bank approves or disapproves the loan. Next, borrower and bank sign an agreement that states all the terms and conditions of the loan including determined interest rate and type of mortgage. Lastly, loan amount will disburse and borrower will start repaying the instalments that constitute principal amount and interest amount for predetermined period of time.\\

And, the major question is how do banks decide whether to give a loan or not? This question is of major concern as bank's cash flow highly depends on timely repayment of the loan. Every bank does not have the same procedure but majority of the loan review process is same. Following are few characteristics that bank officials will concentrate while evaluating a loan application:
\begin{enumerate}
	\item Credit history of applicant
	\item Loan to Value ratio
	\item Employment history
	\item Character assessment of applicant
	\item Evaluation of collateral
	\item Financial statements such as bank history, cash flow, etc. 
\end{enumerate}

\section{Credit Risk}\label{C.risk}
























Credit analysis and assessment is very important for banks and financial instituions to evaluate the credit worthiness of an applicant or a borrower. Banks implements various factors while assessing credit risk; such as credit rating, loan to value ratio, probability of default, etc.; that leads to derivation of credit risk rating. Variety of financial techniques have been used by the officials to analyse credit risk. 

\subsection{Examples of citations}\label{SS.citations}

In \citep{Atiyah:1961,Atiyah:1966a,Atiyah:1966b}, Atiyah builds on the work of \citet{Adams:1962} to develop 
the foundations of topological $K$-theory.  \citet{LewMcG:2000} and \citet{McG:2002} later extend parts of 
this to a previously unexplored algebraic setting.
