%   MSc Business Analytics Dissertation
%
%   Title:     Aaa Bbbbbbb Cccccccccc
%   Author(s): Xxxxxx Xxxxxxxxx and Yyy Yyyyyyyyy
%
%   Chapter 1: Introduction and basic definitions
%
%   Change Control:
%   When     Who   Ver  What
%   -------  ----  ---  --------------------------------------------------------------
%   11Feb11  AB    0.1  Begun
%

\chapter{Introduction}\label{C.intro}
One of the key activities of banking and financial institutions that enhance their quality and financial system, correct handling, and management of liabilities. The performance of those tasks is very crucial for country's economic development, that Irish government witnessed as Irish property bubble that happened in Celtic Tiger period (late 1990 - 2007). While assessing credit risk, it is essential to validate the accuracy and reliability of credit scores or credit rating for all participants. So, How do banks identify a default event: 1. Nonrepayment of the debt to the bank, 2. Repayment is due for more than 90days.\\

This work will discuss predictive models for enhancement in credit analysis and assessment of residential mortgages registered in Ireland using geospatial locations. There have been many studies and researches on how to assess and analyze credit scoring or credit risk, but very few studies are present that describes assessment using geospatial data. This project will demonstrate how geospatial techniques can be used to enhance further credit analyses that empowers banks and financial institutions to take the much better decision on an application. This project will present a predictive model that predicts the probability of default and an interactive visualization highly focused on geospatial locations of residences registered in Ireland and bank's branch locations. The purpose of this visualization is to support decision maker to take a more efficient decision whether to provide loan on a particular house mortgage or not with the use of predicted probability of default. Models for Credit analysis are developed with the use of decision trees using CART algorithm and logistic regression for binary response (dependent) variables. While building models, potential variables were selected based on Information Value statistics. Credibility and quality of the models were evaluated using approaches such as GINI statistics, prediction accuracy, and ROC (Receiver Operating Characteristic) curve.\\

Credit assessment and analysis plays a crucial role in determining the financial strength of businesses and risk estimation that are associated with credit. Following are the primary purposes of assessment of credit:
\begin{enumerate}
\item Helps to keep track of the economy (macro economic perception) 
\item Analyses and ensures stability of financial market (macro prudential perspective)
\item Assessment of quality of collateral/mortgage (monetary policy)
\end{enumerate}

\section{Assumptions \& Challenges}

KPMG provided made up data due to a confidentiality agreement with their client. Data is generated from pre defined formulas that made data look like original real life data, but it could not cover all possible real life scenarios. For example - Data only considers that an applicant will default if it has a credit rating of 5 but data did not consider the situation that a claimant may default if he\/she has a credit score of 2,3,4 and even 1 in some cases. This case depicts a constraint of given data over real life data.\\

Below is a list of assumptions undertaken during the process of practicum:
\begin{enumerate}
\item Property prices have been considered as provided in the data by KPMG; there is no consideration of any time frame. For example, the date when property valuation was done. 
\item Geospatial data such as address latitude and address longitude is assumed to depict geospatial location property correctly.
\item A property is considered as a whole, some apartments and number of floors are ignored. What latitude and longitude of a house consist of 2 floors are same.
\item Dimensions of house and size of the house(number of rooms, bathrooms, lawn, etc.) are not considered during model development.
\item This project only focuses on residential properties, not on commercial properties. 
\item This project did not consider factors such as neighborhood, amenities, and demographics which affects the property price in the market. However, factors such as location, average price have been considered for predicting the probability of default.
\end{enumerate}

\section{Outline}
Below is the flow of the practicum which will give a brief description of each chapter:
\begin{itemize}
\item Business Background \\ This chapter describes business need and contributions in detail. It will explain how this project will contribute towards banks and financial institutions businesses. 
\item Literature Review \\ Chapter 3 presents an in-depth study of academic contributions achieved in the field of credit analysis, geospatial techniques, and data visualization. This section will explain in detail what is credit scoring and what methods have been used in the past to enhance assessment of credit. It will show a comparison between traditional systems and credit scoring along with algorithms to build a model for predicting the probability of default. Later, it will describe geospatial techniques and data visualization techniques.
\item Methodology \\ Chapter 4 will give a detailed explanation of steps and tools that have been used to successfully conduct this project and how different tools have been integrated together.
\item Results \\ Chapter 5 explains the output generated from the methods and algorithms described in the sections mentioned above. It will describe the graphs and images that hold uttermost importance and are relevant to the business need along with Tableau dashboards.
\item Discussion \\ This chapter will discuss data limitations and practicality of the models developed that correctly answers business questions. 
\item Conclusion and Future Work \\ This chapter will conclude the outcome of the practicum along with the improvements and future scope of the project.
\end{itemize}
